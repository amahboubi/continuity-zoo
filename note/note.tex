\documentclass[a4paper,UKenglish,cleveref, autoref, thm-restate]{lipics-v2021}

\bibliographystyle{plainurl}% the mandatory bibstyle

% % \newtheorem{fact}[theorem]{Fact}

% \usepackage[T1]{fontenc} % optional
% \usepackage{amsmath, amssymb, amsthm}
\usepackage{xspace}
\usepackage{macros}
% \usepackage{hyperref}
% \usepackage{cleveref}
%\usepackage{coqtheorem}
\usepackage{tikz-cd}
% \usepackage{adjustbox}
% \usepackage[noadjust]{cite}
\usepackage{mathpartir}
\usepackage{dsfont}
% \usepackage{enumitem}
\usepackage[disable]{todonotes}
\usepackage{csquotes}
% \usepackage{multicol}

% % \newcoqtheorem{theorem}[dummy]{Theorem}
% % \newcoqtheorem{lemma}[dummy]{Lemma}
% \newcoqtheorem{fact}[dummy]{Fact}
% % \newcoqtheorem{corollary}[dummy]{Corollary}

% \crefname{lemmaAux}{Lemma}{Lemmas} %
% \crefname{lemma}{Lemma}{Lemmas} %
% \crefname{corollaryAux}{Corollary}{Corollaries} %
% \crefname{theoremAux}{Theorem}{Theorems} %
% \crefname{factAux}{Fact}{Facts} %
%\setBaseUrl{{https://ps.uni-saarland.de/extras/reducibility-degrees/coqdoc/Undecidability.}}
%\setBaseUrl{{../code-synthetic-reducibility/website/Undecidability.}}

\providecommand{\definedas}{:=}
\newcommand{\lipicsNumber}[1]{\text{{\color{lipicsGray}\sffamily\textbf{#1}}}}

\newcommand{\pto}{\nrightarrow}
\renewcommand{\pto}{\rightharpoonup}
\renewcommand{\pfun}{\rightharpoonup}
\newcommand{\fto}{\rightsquigarrow}
\renewcommand\fto{{\;\mathbin{\clipbox{0pt {-\height} {.24\width} {-\height}}{\ensuremath{\rightsquigarrow}}}{\clipbox{{.72\width} {-\height} 0pt {-\height}}{\ensuremath{\rightharpoonup}}}}\;}

\newcommand\truthtable{\ensuremath{\mathsf{tt}}}

\ccsdesc[500]{Theory of computation~Constructive mathematics}
\ccsdesc[500]{Type theory}
% \ccsdesc[100]{\textcolor{red}{Replace ccsdesc macro with valid one}} %mandatory: Please choose ACM 2012 classifications from https://dl.acm.org/ccs/ccs_flat.cfm 
\Copyright{} %mandatory, please use full first names. LIPIcs license is "CC-BY";  http://creativecommons.org/licenses/by/3.0/

\keywords{type theory, computability theory, constructive reverse mathematics, Coq} %mandatory; please add comma-separated list of keywords

\category{} %optional, e.g. invited paper

%\relatedversion{} %optional, e.g. full version hosted on arXiv, HAL, or other respository/website
% \relatedversiondetails[linktext={xxxx.0000}]{Full version with appendices in main part of paper available on arxiv}{bla} %linktext and cite are optional

% \supplement{{\texttt{\href{https://github.com/uds-psl/coq-synthetic-computability/tree/degrees}{github.com/uds-psl/coq-synthetic-computability/tree/degrees}}}.}%optional, e.g. related research data, source code, ... hosted on a repository like zenodo, figshare, GitHub, ...
%\supplementdetails[linktext={opt. text shown instead of the URL}, cite=DBLP:books/mk/GrayR93, subcategory={Description, Subcategory}, swhid={Software Heritage Identifier}]{General Classification (e.g. Software, Dataset, Model, ...)}{URL to related version} %linktext, cite, and subcategory are optional

%\funding{(Optional) general funding statement \dots}%optional, to capture a funding statement, which applies to all authors. Please enter author specific funding statements as fifth argument of the \author macro.

\nolinenumbers %uncomment to disable line numbering

\hideLIPIcs

\title{Kleene's second algebra over functions and partial functions in constructive type theory}
% \subtitle{Post's Problem for Many-one and Truth-table Reducibility in Coq}

\titlerunning{Notions of continuity in constructive type theory}

%\author{Yannick Forster}{Inria, Gallinette Project-Team, Nantes, France}{yannick.forster@inria.fr}{https://orcid.org/0000-0002-8676-9819}{}%received funding from the European Union’s Horizon 2020 research and innovation programme under the Marie Skłodowska-Curie grant agreement No. 101024493.}

%\authorrunning{Y.~Forster} % mandatory. First: Use abbreviated first/middle names. Second (only in severe cases): Use first author plus 'et al.'

\newcommand{\Baire}{\mathcal{B}}

\begin{document}

\maketitle

\begin{abstract}
  We explain different definitions of continuity in constructive type theory.
  We treat continuity of functions $F : (Q \to A) \to R$,
  i.e.\ with query type $Q$,
  answer type $A$,
  and result type $R$.
  Concretely, we discuss
  modulus continuity (defined by making the list $L \of \List Q$ of queries explicit),
  sequential continuity (defined via tree predicates),
  and forms of
  eloquent continuity (defined via inductive dialogue trees).
  We prove equivalences where possible and isolate necessary and sufficient axioms for equivalence proofs,
  in the spirit of constructive reverse mathematics.  

  % We discuss a definition of Kleene's second algebra in constructive type theory,
  % i.e.\ a partial combinatory algebra (pca) over Baire space $\Baire := \nat \to \nat$.
  % A pca over type $A$ has an application operator $\star \of A \to A \to A$
  % which allows to define operators $K$ and $S$ as in combinatory logic.

  % We then go on to generalise the definitions of continuity to partial functions $F \of (Q \pfun A) \pfun R$
  % and again consider equivalence of the various definitions.
  % We discuss a definition of Kleene's second algebra over partial functions
  % i.e.\ a partial combinatory algebra (pca) over partial Baire space $\mathfrak{B} := \nat \pfun \nat$.
\end{abstract}

\renewcommand\of[1]{: #1}

% \section{Introduction}

\section{Definitions of continuity for total functions}

For the rest of this section we fix a query type $Q$,
answer type $A$,
and result type $O$.
We also fix $F \of (Q \to A) \to R$.

\newcommand\f{f}
\newcommand\g{g}

\subsection{Modulus continuity}

We say that $F$  
has modulus $L \of \List Q$
at $\f : Q \to A$
if
$F$ is determined solely by the queries to $f$ in $L$,
i.e.\ $F \g$ for any $\g$ such that $\forall x \in L.\;  \f x = \g x$ we have $F \f = F \g$.

A function $F$ is \textit{modulus-continuous} if 
\[ \forall \f \of Q \to A.\exists L \of \List Q.\; \forall \g \of Q \to A.\; (\forall x \in L.\;f x = g x) \to F \f = F \g \]

A function $F$ has a
\textit{modulus of continuity function} $M \of (Q \to A) \to \List Q$ if
for all $\f$, $M \f$ is a modulus of continuity for $F \f$.
Formally:
\[ \forall \f \g \of Q \to A.\; (\forall x \in M \f.\;\f x = \g x) \to F \f = F \g \]

\subsection{Eloquent continuity}

We define an inductive type of \textit{dialogue trees} $\mathbb{D}$ with leafs $\eta$ labelled by a result $R$
and internal nodes $\beta$ labelled by a query $q \of Q$ and branching via answers in type $A$:
\begin{mathpar}
  \infer{o \of O}{\eta o \of {\mathbb{D}}}

  \infer{q \of Q \and k \of A \to \mathbb{D}}{\beta q k \of {\mathbb{D}}}
\end{mathpar}

We define an evaluation function $\eval \of {(Q \to A) \to \mathbb{D} \to O}$ as
\begin{align*}
  &\eval \f (\eta o) := o && \eval \f (\beta q k) := \eval \f (k (\f q))
\end{align*}

A function $F$ is \textit{eloquent} via $d : \mathbb{D}$ such that
\[ \forall \f \of Q \to A.\;F \f = \eval \f d \]
and say that $F$ is eloquent if such a $d$ exists.

\begin{lemma}
  Given an eloquency function $d$ for $F$, one can compute a modulus of continuity function $M$ for $F$.
\end{lemma}
\begin{proof}
  Define $M \of {\mathbb{D} \to (Q \to A) \to \List Q}$ by recursion on the tree as
  \begin{align*}
    M (\eta o) \f := [] && M (\beta qk) \f := q :: M (k (f q)) \f 
  \end{align*}
\end{proof}

\subsubsection{Intensional eloquent continuity}
\newcommand\idiag{\mathbb{D}_{\textsf{i}}}

We define the following more intensional variant of continuity.
We define an inductive \textit{predicate} $\idiag : (Q \to A) \to O$ as
%
\begin{mathpar}
  \infer{o \of O}{\eta_{\textsf{i}} o : \idiag (\lambda \f.\; o)}

  \infer{q \of Q \and k \of A \to (Q \to A) \to O \and H : \forall a : A.\; \idiag (k a)}
  {\beta_{\textsf{i}} q k H :  \idiag (\lambda \f.\;k (\f q) \f)}
\end{mathpar}

\begin{lemma}
  If $\idiag F$, then one can construct $d : {\mathbb{D}}$ such that $F$ is eloquent via $d$.
\end{lemma}
\begin{proof}
  We define a function
  $\delta : \forall F' : (Q \to A) \to O.\;\idiag F' \to \mathbb{D}$
  as
  \begin{mathpar}
    \delta F' (\eta_{\textsf{i}} o) := \eta o \and
    \delta F' (\eta_{\textsf{i}} o) := \beta q (\lambda a.\; \delta (k a) (d a))
  \end{mathpar}
  Then 
  $\forall F'.\;\eval \f (\delta F' d) = F' \f$
  follows by induction on $d$.
\end{proof}

\begin{lemma}
  The converse of the lemma is provable assuming functional extensionality.
\end{lemma}

We conjecture that functional extensionality is required here, but do not go into details.

\subsubsection{Brouwer trees}

\subsection{Sequential continuity}

Sequential continuity has a similar intuition than eloquent continuity.
However, instead of defining trees inductively, it relies on an extensional notion of tree predicates:

\newcommand\R{O}

We take any function $\tau \of \List A \to (Q + \R)$ to describe a tree

\newcommand\ask{\mathsf{ask}\;}
\newcommand\return{\mathsf{ret}\;}
We write $\ask q$ for $\mathsf{inl}\; q$
and $\return r$ for $\mathsf{inr}\; r$.

Given a tree $\tau$, we define an operation $\hat\tau : (Q \to A) \to \nat \to Q + \R$ as follows:
%
\begin{mathpar}
  \hat\tau\;f\;(\succN n) := \hat\tau\;(\lambda l.\;\tau (l ++ [f q]))\;\f\;n \text{~~\small if $\tau\;[\;] = \ask q$}
  \and
  \hat\tau\;f\;n := \tau \; [\;]
\end{mathpar}

% \begin{align*}
%   \hat\tau\;f\;0 &:= \None \tag*{\small if $\tau\;[\;] = \None$} \\
%   \hat\tau\;f\;0 &:= \Some (\inr o) \tag*{\small if $\tau\;[\;] = \Some (\inr o)$} \\
%   \hat\tau\;f\;0 &:= \Some (\inl o) \tag*{\small if $\tau\;[\;] = \Some (\inr o)$} \\
%   \hat\tau\;f\;(\succN n) &:= \Some (\inr o) \tag*{\small if $\hat\tau\;f\;n = \Some (\inr o)$} \\
%   \hat\tau\;f\;(\succN n) &:= \Some (\inr o) \tag*{\small if $\hat\tau\;f\;n = \Some (\inl (l, q))$ and $\tau (l \app [f q]) = \Some (\inr o)$} \\
%   \hat\tau\;f\;(\succN n) &:= \Some (\inl (l \app [f q], q') \tag*{\small if $\hat\tau\;f\;n = \Some (\inl (l, q))$ and $\tau (l \app [f q]) = \Some (\inl q')$} 
% \end{align*}

\begin{lemma}
  If $\hat \tau\;f\;n_1 = \return r_1$ and $\hat \tau\;f\;n_2 = \return r_2$ then $r_1 = r_2$.
\end{lemma}

We call a tree $\tau$ well-founded if for any $\f : Q \to A$
there exist $n$ and $o$ s.t.\ $\hat\tau\;f\;n = \return o$.
Thus, every well-founded total tree $\tau$ gives rise to a function $\hat\tau \of (Q \to A) \to R$.

We say that $F$ is \textit{sequentially continuous} if there exists a tree $\tau : \List A \to Q + R$
such that for % every $i$, $\tau_i$ is well-founded and total and
$\forall f.\;\exists n.\; \hat \tau\;f\;n = \return (F f)$.

\begin{lemma}
  If $F$ is eloquent, $F$ is sequentially continuous.
\end{lemma}
\begin{proof}
  We construct a function $T$ from dialogue trees to well-founded total trees:
  \begin{mathpar}
    T (\eta o) l := \return o \and
    T (\beta q k) [] := \ask q \and
    T (\beta q k) (l \app [a]) := T (k a) l
  \end{mathpar}

  The equation $\eval\;d\;f = \hat{T d} f$ follows by induction on $d$.
  % In the second case, we need to prove that
  % $\hat{T (\beta q k)}\; f\; (1 + m) = \Some (\inr (\eval\;f\;(k (f q))))$
  % given that
  % $\hat{T (k (f q))}\; f\; m = \Some (\inr (\eval\;f\;(k (f q))))$.

  % This follows with $n_1 = 0$ from the following suitably generalised lemma to be proved by induction on $n_2$:

  % \itshape
  % If  $\hat \tau\; f\; n_1 = \Some (\inl (l, q))$,
  % $\tau_2 l' := \tau (l ++ [f q] ++ l')$.
  % Then there are three cases:
  % \begin{enumerate}
  % \item $\hat {\tau_2} f n_2 = \Some (\inr o)$ and $\hat \tau\; f\; (1 + n_1 + n_2) = \Some (\inr o)$,
  % \item $\hat {\tau_2} f n_2 = \Some (\inl (l', q'))$ and $\hat \tau\; f\; (1 + n_1 + n_2) = \Some (\inl (l ++ [f q] ++ l', q'))$,
  % \item $\hat {\tau_2} f n_2 = \None$ and $\hat \tau\; f\; (1 + n_1 + n_2)$.
  % \end{enumerate}
\end{proof}

\section{Self-modulating Moduli}

In this section, we show that sequential continuity is equivalent
to having a modulus of continuity function which forms is also a modulus function for itself,
a result due to Fujiwara and Kawai.
The backwards direction of the equivalence crucially only works for $Q = \nat$,
and then on the availability of $\Delta^0_1\textsf{-AC}_{X,\nat}$ for any type $X$.
This choice principle is available in all constructive foundations.

\begin{lemma}
  If $F$ is sequentially continuous it has a continuous modulus of continuity function.
\end{lemma}
\begin{proof}
  Let $\tau$ witness sequential continuity of $F$.
  We construct a function $M' \of (Q \to A) \to \nat \to \List Q \times (Q + \R)$ similar to the equations of $\hat \tau$, but threading the list of queries around $l$.
  \begin{mathpar}
    M'\;f\;(\succN n) := q :: M'\;(\lambda l.\;\tau (l \app [f q]))\;\f\;n \text{~~\small if $\tau\;[\;] = \ask q$}
    \and
    M'\;f\;n := []
  \end{mathpar}
  Note that since
  $\forall \f.\exists n.\exists r.\;\hat\tau\;\f\;n = \return r$
  we can compute this $n$ since the condition is decidable.
  let $N : (Q \to A) \to \nat$ be the resulting choice function
  and define
  $M f := M'\;\f\;(N \f)$,
  which is a modulus of continuity for $F$ and continuous itself.
\end{proof}

We are now going to prove that for the special case $Q = \nat$ one can turn
any continuous modulus of continuity function $M$ for a given $F$ into a self-modulating
modulus of continuity function $N$ for $F$.
For $Q = \nat$, we can use the alternative definition of continuity where the returned modulus is just a natural number rather than a list of questions.

Ideally, we would then like to define $N f := \mu k.\;\forall g.\;(\forall m < k.\;f m = g m) \to F f = F g$.
However, the condition on the right is not $\Sigma^0_1$, and thus the minimisation is not computable.
Minimality of $k$ is however not important:
it suffices to pick any unique $k$ respecting extensionality of $f$.
\todo{Assia, is this true? We use minimality in the proof, but could we relax it somewhat?}

We follow Fujiwara and Kawai and define $N f := \mu n.\;M (f \downharpoonright n) < n$
where $f \downharpoonright n$ is the function that behaves like $f$ on input $m < n$, and returns a fixed dummy element $a_0$ otherwise.

\begin{lemma}
  If $Q = \nat$, $a_0 : A$, and $F$ has a continuous modulus of continuity function,
  then $F$ has a self-modulating modulus of continuity function.
\end{lemma}
\begin{proof}
  Let $M$ be a continuous modulus of continuity function for $F$.
  Note that we can assume $M : (\nat \to A) \to \nat$ with
  \[\forall f g.\; (\forall m < M f.\;f m = g m) \to F f = F g\]
  because we are working with $Q = \nat$.

  First observe that since $M$ is continuous, we can prove
  \[ \forall f.\;\exists n.\; M (f \downharpoonright n) < n \]
  where $f \downharpoonright n$ is the function that behaves like $f$ on input $m < n$, and returns $a_0$ otherwise.
  The proof is straightforward: Given $f$, use continuity of $M$ on $f$ to obtain a modulus $m : \nat$, then pick
  pick $1 + M f + n$.
  
  Since $<$ is decidable, we can define a function $N$ returning the \textit{least} $n$ such that this condition holds.
  We call the property that $N$ returns one such $n$ its defining property, and that its the least such $n$ its minimality property.
  
  We first prove that $N$ is self-modulating.
  Let $f$ and $g$ with $(*) : \forall n < N f.\; f n = g n$ be given.
  We have to prove $N f = N g$.
  We show $N f \geq N g$ and $N f \leq N g$.

  For the first,
  we use the minimality property of $N$ and
  $N (g \downharpoonright N f) =
  N (g \downharpoonright N f) < N f$,
  where the equality is by $(*)$ and the inequality by the defining property of $N$.

  For the second, we may additionally assume $N g < N f$.
  We apply the minimality property of $N$ again and have
  to prove $N (f \downharpoonright N g) < N g$,
  which by $(*)$ and $N g < N f$ is equivalent to $N (f \downharpoonright N g) < N g$, which follows by the defining property of $N$ again.

  TODO
\end{proof}

Note that generalising beyond $Q = \nat$ seems infeasible.
One route to go would be to generalise to any retract of $\nat$,
but this only extends the result to finite types,
where any function with finite query type is continuous.

Other generalisations, at least of the present proof, seem not possible:
First, turning a list of queries and answers into a function requires decidable equality on $Q$.
Secondly, the necessary choice function has to return minimal elements,
meaning we would either have to require a size function into natural numbers and a minimising choice principle w.r.t.\ this size,
or a decidable, antisymmetric, linear order on $Q$ with a minimising choice principle.
Both properties seem rather strong and possibly are equivalent to $Q$ being a retract of $\nat$ again.  

\section{Eloquent continuity, sequential continuity, and (generalised) bar induction}

\newcommand\CI{\mathsf{CI}}
\newcommand\GBI{\mathsf{GBI}}
\newcommand\BI{\mathsf{BI}}

In this section we deal with the statement that sequentially continuous functions $F : (Q \to A) \to \R$ are eloquent.
Since this principle allows turning an extensional tree into an inductive tree,
we call this principle $\CI(Q,A,R)$ (continuity induction).
The notion is inspired by Brede and Herbelin's generalised bar induction principle $\GBI(Q,A)$.
$\GBI(Q,A)$ states that bars with internal nodes labeled with $Q$ and branching over $A$
which are barred are in fact inductively barred.
A bar is the complement of a tree,
and the inverse of a barred bar is a well-founded tree.

They prove that $\GBI(\nat,A)$ is equivalent to regular bar induction $\BI(A)$,
where the nodes of the bars are unlabeled and they still branch via $A$.

We extend their contribution proving that in fact this equivalence preserves the logical complexity of
trees under mild assumption on the complexity class.

We then prove that $\Delta^0_1\text{-}\GBI(Q,A)$ implies $\CI(Q,A)$,
and $\CI(\nat,A)$ implies $\Delta^0_1\text{-}\BI(A)$.
Since $\BI$ deals with unlabeled trees,
we here need to use the equivalence between eloquent continuity via dialogue and via Brouwer trees.

\begin{lemma}
  $C\text{-}\GBI(\nat,A) \leftrightarrow C\text{-}\BI(A)$,
  for $C$ being a class of predicates, i.e.\ $C : \forall X.\;(X \to \mathbb{P}) \to \mathbb{P}$
  closed under TODO.
\end{lemma}

\begin{lemma}
  $\GBI(Q,A) \to \CI(Q,A,R)$
\end{lemma}
\begin{proof}
  TODO
\end{proof}

\textbf{Note that the following proof is not a proof, just a wish, I have not checked the details and just outlined how a really nice proof could look:}

\begin{lemma}
  $\CI(\nat,A,\List A) \to \BI(A)$.
\end{lemma}
\begin{proof}
  Via the equivalence proved before, we assume that any sequentially continuous function over query type $\nat$
  is eloquent via a Brouwer tree.

  Let a bar $\tau : \List A \to \bool$ be given.
  Assume that $\tau$ is barred, i.e.\ that
  $\forall \f : (\nat \to A).\exists n.\;\tau [f0,\dots,fn]$.

  We first turn $\tau$ into a valid computable tree $\tau' : \List A \to Q + \R$.
  $\tau' l$ searches for the shortest prefix $l'$ of $l$ such that $\tau l'$,
  and if this exists returns $\return (\iota l)$.
  or otherwise asks the length of $l$ as question.

  Since $\tau$ is barred, $\tau'$ is well-founded.
  Now $\tau$ computes the function $F : (Q \to A) \to \List A$
  which on input $f$ uses $\Delta^0_1\text{-AC}_{\nat \to A, \nat}$ to extract the natural number from the barredness assumption.  
  Consequently, $F$ is sequentially continuous.

  By assumption, we obtain a Brouwer tree computing $F$.
  By induction on this Brouwer tree, we show that $\tau$ is inductively barred.
\end{proof}

\bibliography{biblio.bib}

\end{document}
